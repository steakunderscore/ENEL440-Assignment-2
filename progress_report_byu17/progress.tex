\documentclass[11pt]{article} 

% Simpler tables
\usepackage{tabularx}

% Muliple rows
\usepackage{multirow}

% Margins
\usepackage[margin=2.5cm]{geometry}

% Images
\usepackage{graphicx}

% Basic math stuff.
\usepackage{mathtools}
\usepackage{amsmath}

% Allows us to use .eps files.
% pdflatex --shell-escape --synctex=1 file.tex
\usepackage{epstopdf}

% Images in figures.
\usepackage{subfig}

% Colours
\usepackage[usenames,dvipsnames]{color}
\definecolor{MyLightGray}{gray}{.90}

% Font stuff
\usepackage{palatino} % A nice alternative to Computer Modern

% Hyperlinks
\usepackage{hyperref}

% Code listings
\usepackage{listings}
\lstset{
        frame=single,
    basicstyle=\small\ttfamily,
        keywordstyle=\color{black}\bfseries,
        commentstyle=\color{OliveGreen},
}

% Tightly spaced lists, example:
% \begin{compactitem}
% \item ...
% \end{compactitem}
\usepackage{paralist}

\begin{document}
\title{Particle Filter Techniques and Applications}
\author{Benjamin Washington-Yule, Henry Jenkins}
\maketitle

\begin{abstract}
Probably don't need an abstract for a 4 page progress report.
\end{abstract}

\section{Introduction}
The report briefly summarizes the progress made to date of the second ENEL440
assignment. Much of the progress has focus on acquiring an in-depth knowledge
of the allocated subject, and has required a large amount of reading. Particle
filtering relies on rather complex ideas and it is important to develop a good
understanding of these before attempting to develop an application.

Even so, a suitable application that is already commonly used for a closely
related filter (the Kalman filter, see below) has been decided on and is
discusses in section \ref{sec:application}.

\section{Background}
Particle filters are a class of sophisticated sequential estimation techniques
for estimating the state of a dynamical system. They are sequential in that they
incrementally approximate the probability distributions of a system as more
information is gathered. The applications of particle filters are often the
same as those of the closely related Kalman filter, among which the most common
are: optical tracking, communications systems and robotics sensors. However,
where the Kalman filter estimation technique fails when the noise is not
Gaussian or the system is non-linear, particle filter techniques may potentially
succeed.

\subsection{Sequential Estimation}
The sequential nature of the particle filter method is important. It is
generally desirable to use information incrementally as it arrives, rather
than gathering all of the required information and then finally using it to
estimate the statistics of a system. 

A simple example is a physical process that is described by a random variable
for which we have no knowledge of the statistics. The simplest statistic we may
wish to estimate is the mean. If we did indeed have the PDF it could find it
using
\begin{equation} \label{eq:someequation}
\mu = \int{xp_{X}(x)\textrm\,{d}x},
\end{equation}
however we have assumed that we do not have such knowledge. We can still
estimate the mean by collecting samples of the process $\{x_{1}, x_{2}, ... ,
x_{n}\}$. The sample mean is then given by
\begin{equation}
m = \frac{1}{n}\sum\limits_{i = 0}^{n}{x_{i}}
\end{equation}
This is not altogether desirable because we must wait until all $n$ observations
have been made before using the information. It would be better if we could
incrementally update our estimate of the mean as more information is acquired.
In the case that we only have one sample $x_{1}$, our best estimate of the mean
would be the sample itself, i.e.,
\begin{equation*}
m_{1} = x_{1}
\end{equation*}
Extending this to $k$ samples, the current best estimate can be written as a
partial sum,
\begin{equation}
m_{k} = \frac{1}{k}\sum\limits_{i = 1}^k{x_{k}}
\end{equation}
Since we wish to sequentially update our estimate, our goal is to write this
in terms of our current observation $x_{k}$ and the previous estimate of the
mean $m_{k-1}$,
\begin{align*}
m_{k} &= \frac{1}{k}x_{k} + \frac{1}{k}\sum\limits_{i = 1}^{k-1}{x_{k}} \\
&= \frac{1}{k}x_{k} + \frac{1}{k}\frac{k-1}{k-1}\sum\limits_{i=1}^{k-1}{x_{k}}
\\
&= \frac{1}{k}x_{k} + \frac{k-1}{k}\frac{1}{k-1}\sum\limits_{i=1}^{k-1}{x_{k}}
\\
&= \frac{1}{k}x_{k} + \frac{k-1}{k}m_{k-1}
\end{align*}
This is the sequential estimate of the mean of our simple random variable.

\subsection{State Space}
We can extend sequential estimation ideas above to the state of a dynamic
system. Let $\mathbf{x}_{k}$ denote the state of the system at time $k$,
$\mathbf{u}_{k}$ denote the input to the system at time $k$ and
$\mathbf{z}_{k}$ denote the observation at time $k$. We form the usual state
space representation of the system,
\begin{equation}
\mathbf{x}_{k} = \mathbf{A}\mathbf{x}_{k-1} + \mathbf{B}\mathbf{u}_{k} + \nu
\end{equation}
\begin{equation}
\mathbf{z}_{k} = \mathbf{C}\mathbf{x}_{k} + \eta
\end{equation}
It is worth noting that by written the state space thus we have assumed
linearity.

The Kalman filter attempts to estimate the current state $\mathbf{x}_{k}$ using
the distribution $\textrm{P}(\mathbf{x}_{k}\,|\,\mathbf{z}_{0:k})$, that is, the
unobserved state given the sequence of observations up to time $k$, which
it assumes it is a multivariate Gaussian distribution. The particle filter does
not make this assumption, and this means that
$\textrm{P}(\mathbf{x}_{k}|\mathbf{z}_{0:k})$ cannot be written in a simple
form~{\cite{ref:1}}. So we instead approximate the distributions
by discrete random measures defined by particles $x^{(m)}$, each with an
associated weight $w^{(m)}$, where $1 < m < M$~{\cite{ref:2}}.

We can use the particles and corresponding weights to approximate a general
distribution $p_{X}(x)$ by
\begin{equation}
p_{X}(x) ~= \sum\limits_{m=1}^{M}{w^{(m)}\delta(x-x^{(m)})}
\end{equation}
Using this approximation allows calculation of statistics of a system to be
simplified to summations, as opposed to expressions involving possibly complex
integrals. 

But this begs the question, how were the set of particles that approximate the
distribution created? One answer is \emph{importance sampling}, i.e., the
particle filter is simply acting as the importance sampler.

\subsection{Importance Sampling}
Importance sampling allows the approximation of a distribution $p_{X}(x)$ with
a discrete random measure. It involves drawing $M$ particles from a
distribution $\pi(x)$ known as the \emph{proposal distribution}. The
weights of the particles at each sample point are calculated using
\begin{equation} \label{eq:weights}
w^{(m)}_{k} \propto
\frac
{p(z_{k}\,|\,x^{(m)}_{k})\,p(x^{m}_{k}\,|\,x^{(m)}_{k-1})}{\pi(x^{(m)}_k\,|\,x^{
(m)}_{0:k-1}, z_{0:k}}
\end{equation}
The particles are then \emph{re-sampled} at each time step. The importance
sampling algorithm is thus implemented using two steps for every $k$: (i) Draw
the particles from the proposal distribution and (ii) computing the weights
according to \ref{eq:weights}, where the initial weights are simply
$\frac{1}{N}$ where $N$ is the number of particles drawn in step (i).

\subsection{Summary}
A particle filter extends the situations where system state can be estimated. A
Kalman filter can be used to estimate the state of the system above, but only
because it is linear, and the noise is (assumed) Gaussian. The particle filter
method can deal with non-linear systems so we can extend the general state
space thus:
\begin{equation}
\mathbf{x}_{k} = f(\mathbf{x}_{k-1}, u_{k-1}) + \nu
\end{equation}
\begin{equation}
\mathbf{z}_{k} = g(\mathbf{x}_{k}) + \eta
\end{equation}
Where $f$ and $g$ are general functions that are non-necessarily linear and
neither noises $\nu$ or $\eta$ are necessarily Gaussian.

The particle filter has much the same applications as the extended Kalman
filter. In the case that the state space is linear and the noise Gaussian, the
Kalman and particle filters theoretically achieve the same results. It is in
the case where the state space is non-linear and/or the noise not Gaussian that
the usefulness of the particle filter method becomes apparent.

\section{Application Example}\label{sec:application}
The application chosen is video tracking, that is, tracking an object on screen.
This is a common application of a Kalman filter and it is likely that a Kalman
(or extended Kalman) filter would do a sufficient job, but the goal is to do the
same thing with a particle filter and if possible, compare the results to that
of a Kalman filter.

No single reference video has been decided on, this is intentional as we are
unsure at this stage of the difficulty of the code given the time constraint. A
2D video simple objects will be the easiest to track, but a real-life 3D video
obviously has more applications. It is hoped that even if the particle filter
method is only implemented for a 2D video, it will be left in such a state that
it can with a little more work, handle more complex video.

Looking at video tracking in the simplest way possible, it is the association of
an object in consecutive video frames. It is interesting that this is a task
for which the human eye is very well suited, but can be very difficult for
computers. 

\subsection{Plan}
The application chosen to illustrate our topic is video tracking, as discussed
above. We intend to implement code in Matlab to track objects in videos to (i)
prove that particle filtering techniques can be used for this application,
(ii) analyse the performance and usefulness of these techniques and (iii)
compare the performance with the Kalman filtering method. A metric for success
has not been decided upon but will hopefully become clearer after some
experimentation.

The videos chosen will ideally feature a single object moving inside some
stationary scene. If we are able to successfully track the object then it would
be interesting to investigate the tracking performance when the object is
obscured (behind some environmental feature) for one or more frames. After a
little investigation it is clear that there is a large amount of videos that
would satisfy our requirements on the internet.

\section{Math}

% Bibliography. The "99" allows up to 99 entries apparently.
\begin{thebibliography}{99}
\bibitem{ref:1} S. Arulampalam, S. Maskell, N. Gordon, and T. Clapp. A tutorial
on particle filters for on-line non-linear/non-gaussian baysian tracking.
\emph{IEEE Transactions on Signal Processing, 50(2):174-188, February 2002.}

\bibitem{ref:2} P. M. Djuric, J. H. Kotecha, J.Zhang, Y. Huang, T. Ghirmai, M.
F. Bugallo, J. Miguez. Particle Filtering. \emph{IEEE Transactions on Signal
Processing, 20(5):19-37, September 2003.}
\end{thebibliography}

\end{document}