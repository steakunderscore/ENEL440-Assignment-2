\section{Introduction}
The use of particle filters in signal processing is a comparatively new area of
research \cite{ref:1}. The particle filter is a type of \emph{recursive Bayesian estimator},
also known as a \emph{Bayes filter}, and is closely related to another type
of Bayes filter called the \emph{Kalman filter}. Bayes filters find
applications in a wide range of areas, which includes among other \cite{ref:2},
\begin{compactitem}
\item Robotics, to infer position and orientation.
\item Computer vision, to track objects.
\item Medical fields, to extract information from noisy data.
\end{compactitem}
A Bayes filter is used in sensor and data fusion. It allows sensor measurements
to be combined with a state model to accurately track the state of a system.

The particle filter differs from the Kalman in the assumptions that are made
about the underlying system model and sensor noise. A more rigorous discussion
of these differences is found in Section \ref{sec:background} but the particle filter
can be generally described as having looser requirements, i.e., fewer
assumptions must be made about the model and sensor noise for a particle filter.
This may mean that in some cases a particle filter will succeed where a Kalman fails.