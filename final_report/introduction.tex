\section{Introduction}
The use of particle filters in signal processing is a new area of ongoing
research. The particle filter is a type of Bayes filter, and is closely related
to another type of Bayes filter, the Kalman filter. Particle filters find
applications in similar areas as the Kalman filter, that is:
\begin{compactitem}
\item TODO
\item TODO
\end{compactitem}
The particle filter differs from the Kalman in the assumptions that are made
about the underlying system model and sensor noise. A more rigorous discussion
of these differences is found in Section \ref{sec:TODO} but can be generally
described as having looser requirements, i.e., fewer assumptions need to be made
about the model and sensor noise for a particle filter. This may mean that in
some cases a particle filter will succeed where a Kalman fails.