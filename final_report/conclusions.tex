\section{Conclusions}
%TODO
The results observed for both application examples are in line with what was
expected, with the exception of the Kalman filter implementation's not failing
when presented with a non-linear system model.

The image tracking application was successful despite the overly simplistic 
model used. However, it is important not to draw unwarranted conclusions 
from this example and assume that an
inaccurate system model will not affect the Bayes filter's ability to track an
image. On the contrary, the results show that an inaccurate model is
more detrimental than high measurement noise. If the first application example 
was even slightly
more complex, e.g., giving the ball a second degree of freedom, then our
overly simplistic model that only takes gravity into account would certainly not
work as intended.

The second filtering application example showed that we are able to obtain very accurate tracking 
(filtering) even in the presence of atypically high sensor noise, but that this
is only because we can characterize the noise accurately. We conclude that 
recursive Bayesian estimation is an extremely powerful tool, but that its power
depends on
\begin{compactitem}
\item The accuracy of the sensor model.
\item The accuracy of the dynamical model.
\end{compactitem}
i.e., a Bayes filter is only as accurate as the model and the characterization 
of the noise.

If either of these are below a certain threshold (dependent on the situation) 
then the other must be very accurate to compensate or tracking will be lost
entirely.

\subsection{Particle Filter Implementation}
We conclude that the particle filter implementation of the Bayes filter is 
difficult to implement at best, especially in Matlab. We come to this conclusion
after observing the following:
\begin{compactitem}
\item There is very little reference material on the internet about particle filters
in software, the few that do exist are implemented in C.
\item Our Matlab implementation did not appear to work.
\item Many papers that mention particle filtering, in fact write the implementation
using an alternate (usually Kalman) implementation.
\end{compactitem}
It should be pointed out that we are unsure whether our particle filter implementation
was close to, or far from functioning. Given more time, it would be desirable to 
investigate this further.

Even without a working implementation, we have still built the framework for a
particle filter implementation. The system, dynamical and noise models are
all complete, and with further work, the particle filter predict and update
routines could easily be inserted into the code.

From all evidence we conclude that for all examples presented in this report, 
the particle filter implementation would produce results equal to or better than
our test Kalman implementation.

One point to note about the application described in Section \ref{sec:image-tracking}
is that if were we to employ a more accurate model for the balls motion, e.g., 
a model that takes into account the impulse from hitting the ground and a sudden change in velocity,
then it would likely become a very non-linear system which may eventually 
cause the Kalman implementation may break down, and a particle filter become necessary.
However, we have also observed that the Kalman filter can handle non-linearities so 
the particular threshold is unknown.

It appears that for most applications, the Kalman filter performs 
satisfactorily, even in the presence of slight non-linearity in the
system model. However, we assume that there exist models of sufficient complexity
that will benefit from the particle filter method.
