
\begin{abstract}
This report describes the general Bayes filter and discusses two implementations
thereof: the Kalman filter and the particle filter. The framework for the Bayes filter is discussed
in detail, including the formulation of a state space model for a system. Two
application examples are chosen and simulated: image tracking and noise filtering.

The results for each are in line with expectations: the filter performance depends
on the accuracy of the model and the amount of measurement noise. Only the Kalman
filter implementation is functional, and it is not known why the particle filter
routine does not work. The implementation of the particle filter is examined
and compared to the Kalman implementation from a theoretical viewpoint. 

The results of the second example show that the Bayes filter can handle systems
where the noise is very high. It concludes with a short discussion where it is
concluded that the Kalman filter is sufficient for the majority of situations.
\end{abstract}
